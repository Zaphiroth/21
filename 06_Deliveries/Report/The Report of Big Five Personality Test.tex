\documentclass[UTF8]{ctexart}
\usepackage{hyperref}
\hypersetup{
  colorlinks=true,
  linkcolor=blue,
  filecolor=blue,
  urlcolor=blue,
  citecolor=cyan
}
\pagestyle{plain}
\begin{document}
  \title{关于大五人格测验的探索性分析报告}
  \author{刘哲}
  \maketitle
  \part{背景介绍}
    \section{人格测验}
      世界上没有两片相同的树叶,同样,世界上也没有完全相同的两个人。
      与树叶不同,人与人之间的差异不光体现在外表上,还更多的体现在人格上。
      俗话说,“画虎画皮难画骨,知人知面不知心”,对于一个人的外表如何,我们很容易就能形容出来;
      但是对于一个人的人格如何,首先单薄的语言很难准确描述人格特点,
      其次评价一个人是非常主观的事,“一千个人有一千个哈姆雷特”,没有统一的评价标准。
      因此,为了满足个人以及社会对于了解人格的需要,尤其是心理学对人格研究的需要,
      出现了用于测试个人行为独特性和倾向性等特征的测验形式,称为人格测验(personality test)。\par
      % 人格测验一般采用问卷的方法进行,通过设置一定的场景,引导测验人回答一系列精心设计的问题——
      % 一般为没有褒贬的中性问题,将个人的思维方式和行为方式具象化地映射出来,
      % 随后可以将个人的各种特征量化成一系列多维度的评分,据此划分出不同的人格类型,
      % 或直接以每个人的评分作为人格倾向的测验结果。\par
      人格测验为参与回答同一份测试问卷的人提供了一个统一的人格评价标准。
      尽管一次人格测验的结果受到测验人当时心理状态的影响比较大,
      但也可以通过精心设计问卷题目,或者取多次测验的平均结果,得到一个比较稳健的人格评价。
      又因为有着评价客观、解释容易、可操作性强的优点,
      除心理学研究外,人格测验也广泛应用于公司的人事选拔中。\par
      得益于网络的普及,任何人都可以很容易地获得一份人格测验的问卷,
      花上十分钟填写就能得到一份关于自己的人格报告,人格测验的衍生种类也变得越来越五花八门。
      根据心理学上对于人格的假说不同,大体可归为“类型说”和“特质说”两类。
      前者认为所有人可划分成特定的几种类型,并表现出与对应类型相似的行为模式,
      典型代表是广为人知的 MBTI 十六型人格(Myers-Briggs Type Indicator);
      后者则认为人的行为模式是由一种心理结构引导的,不同的刺激经过一系列运作,
      在引发行为上可能具有同等作用,并且产生对应的适用性和表现性行为,
      典型代表就是大五人格理论(Big Five Personality Traits)。
    \section{大五人格理论}
      大五人格理论是人格心理学中“特质流派”的典型代表,发端于人格词汇学研究。
      将词典中描述性格人格特质的词语进行汇总、归类、概括,几经发展,最终形成了五因素人格,
      即用五个关键词概括人格特质,分别为开放性(Openness)、尽责性(Conscientiousness)、
      外倾性(Extraversion)、宜人性(Agreeableness)、情绪性(Neuroticism)。\par
      大五人格理论有其独特的科学性。首先,五个关键词来源于词典,即人类用于描述人格特质的形容词。
      其次,多个独立心理学研究都发现了人格特质中五个最主要因素的存在。
      再次,人格的形成极其复杂,大五人格理论没有将人格分成特定的类型,而是采用倾向性分数的形式,
      既能比较全面稳定地描述人格,又不会产生过多数量的类型划分。
      最后,五个维度更容易解释和让人接受,可能是能比较好地描述人格的最低维度,
      著名的卡特尔十六型人格测验(16PF)的作者卡特尔教授也认为,
      其16个维度继续降维就是与五因素人格非常相似的5个维度。
    \section{大五人格测验}
      常用的大五人格测验题目有 10-item scale 和 20-item scale
      (\href{https://ipip.ori.org/newBigFive5broadKey.htm}{Big-Five Factor Markers}),
      分别对应每个维度10个题目和20个题目,采用 five-point scale 的形式,
      以1$\sim$5表示符合程度,其中“1”表示完全不符合,“3”表示中立,“5”表示完全符合,
      最后分别统计每个维度上的分数,分数高低代表参加者更倾向于该维度的哪一端,即有某种特质表现的可能性。
      \begin{itemize}
        \item Openness,开放性:指个体对经验持开放、探求的态度。
        得分高者不墨守成规,更倾向于独立思考;得分低者比较本分,更喜欢实干。
        \item Conscientiousness,尽责性:指个体在目标导向行为上的组织、坚持和动机。
        得分高者做事有条理,并能持之以恒;得分低者马虎大意,容易见异思迁,不可靠。
        \item Extraversion,外倾性:指个体对外部世界的积极投入程度。
        得分高者热爱交际,精力充沛、乐观、友好;得分低者更加谨慎、冷静,喜欢独处,少说多做。
        \item Agreeableness,宜人性:指个体在合作与社会和谐性方面的差异。
        得分高者富有同情心,更注重合作;得分低者喜欢为了自己的利益和信念而奋斗。
        \item Neuroticism,情绪性:指个体体验消极情绪的倾向。
        得分高者更容易烦恼和焦躁,出现情绪化反应;得分低者擅长自我调节,不易出现极端反应。
      \end{itemize}
      \par
      因此,大五人格模型也被称为 OCEAN 模型。
      必须强调,五个维度上的得分高低只能代表特质的倾向,不代表人格的优劣。
  \part{数据描述}
    \setcounter{section}{0}
    \section{研究目的与数据来源}
      前面提到,大五人格测验的结果是五个维度上代表特质倾向的分数,
      虽然它很好地将“人格”这一概念量化了,但没有给出明确的高低界限,也不便于形成浅显易懂的文字描述。
      一个维度上高于多少分才算高分者,低于多少分才算低分者?
      如何将量化的特质得分准确转换成恰当程度的词汇描述?都有很大的解释空间。\par
      本文采用的数据为在线人格测验网站\href{https://openpsychometrics.org}{Open Psychometrics}
      于2016$\sim$2018年间收集到的测验数据,为 10-item scale 版,且参加者均同意其用于科学研究。\par
      原始数据共包含1015341条测验结果;除对应50个问题的变量外,还有一些参加者的个人信息,
      考虑到数据的准确性和可用性,这部分仅保留参加者的国家信息,用于研究国家之间人格特质的差异。
    \section{数据清洗和数据预处理}
      \subsection*{删除无效数据}
        首先去除数据中的空行,这可能是由数据收集过程中的失误造成的,共删去??条数据。
        其次,由于测验问卷没有要求和检验每道题目的回答必须不为空,数据中存在部分记录为“0”的缺失值需要处理。
        对于缺失值采用两种处理方法:因为每个维度下仅有10个题目,
        认为全部题目中缺失大于等于6个回答将造成比较严重的结果偏差,所以将其作为无效数据删去;
        缺失回答小于等于5个的情况,则对其进行补数。\par
        去除无效数据后得到1008190条有效数据,其中含有缺失值的数据有??条。
      \subsection*{填补缺失值}
        同一维度下的10个问题具有一定同质性,具有相似人格特质的人在同一维度下很可能会有相似的回答,
        因此,采用K最近邻方法补数是合理的。\par
        具体做法为:取含缺失变量数据的正常变量部分,计算其与完整数据对应变量之间的欧氏距离;
        选择在欧氏距离下与含缺失变量数据最相近的10条完整数据,根据距离确定权重,距离越近,权重越高;
        对缺失变量计算加权平均,作为缺失值的填补。\par
        其中有三点需要注意:一是每个变量的取值均为严格的1$\sim$5,因此计算距离前不需要进行数据标准化;
        二是加权平均的结果会有小数,需要将其四舍五入为整数1$\sim$5;
        三是四舍五入会导致缺失值填补为1和5的可能性低于填补为2、3、4的可能性,
        大五人格理论假定大多数人都在人格维度中处于平均值,出现极端性格的可能性较低,
        再加上缺失值比例较低,因此认为该误差可忽略。
      \subsection*{数据预处理}
        经过数据清洗,最后得到1008190条样本数据。为了便于分析,需要对数据进行一些处理。\par
        首先,大五人格测验中的一些题目是反向记分的,要将这些题目的记分由1$\sim$5改为5$\sim$1。
        之后,在进行数据挖掘时,先将数据中心化,即现有分数全部减3,使分数体系变为-2$\sim$2。
        根据大五人格理论的假定,这样每个变量应近似服从均值(中位数)为零的分布。
    \section{数据可视化展示}
\end{document}